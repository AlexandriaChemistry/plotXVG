\clearpage
\section{Using plotXVG}
The general command for using plotXVG is:
\begin{verbatim}
    plotxvg -f file(s) [-flags]
\end{verbatim}
As apparent, there is only one required argument, namely an input file in .xvg or .csv format. This will produce a scatter plot from your input data. An external window will automatically be opened by matplotlib showing you an interactive plot from which you can save it by the supported buttons in the interactive window. Alternatively, plotXVG has a \texttt{-save} flag which saves the plot after the layout of the plot parts has been regulated. To clarify, the interactive matplotlib window might not show the actual finalized image. If no external window is wished to be opened, \texttt{-noshow} can be added to the command.
\begin{verbatim}
    plotxvg -f file(s) -save output.pdf -noshow
\end{verbatim}
In order to create for example a line plot, one additional argument \texttt{-ls} needs to be added along with a line style in order to tell plotXVG that lines rather than markers are desired.
\begin{verbatim}
    plotxvg -f file(s) -ls solid
\end{verbatim}
will create a line plot with a solid line. Supported line style are solid, dashed, dotdashed and dotted lines. The user can also by all means combine lines with markers by adding both flags:
\begin{verbatim}
    plotxvg -f file(s) -ls dashed -mk +
\end{verbatim}
which results in a dashed line with crosses on each data point.